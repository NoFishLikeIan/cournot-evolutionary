\documentclass[american]{scrartcl}
    \usepackage{babel}
    \usepackage[utf8]{inputenc} 
    \usepackage{csquotes}
    \usepackage{amsmath}
    \usepackage{amssymb}
    \usepackage{graphicx}   
    \usepackage{mathtools}

    
    \setlength{\parindent}{0em}
    \setlength{\parskip}{0.5em}

    
    \title{Automata notes - Evolutionary game theory}


    \author{Andrea Titton}
    

% Commands
\newcommand{\set}[1]{\left\{#1\right\}}
\newcommand{\Real}{\mathbb{R}}
\newcommand{\abs}[1]{\left\lvert #1 \right\rvert}
\newcommand{\Two}{\mathbf{2}}

\begin{document}

% Title

\maketitle

\section{Notes}

We can represent a strategy over $m$ states and $n$ actions as a pair $(\Sigma, X)$ where,

\begin{equation}
    \Sigma \in \Two^{m \times n}, \ X \in \Two^{m \times n \times m}
\end{equation}

A game can be played between two strategies as long as the number of actions in the same.

Assume two strategies $(\Sigma_g, X_g)$ and $(\Sigma_y, X_y)$, with a different number of states, $m_g$ and $m_y$. A state can be represented with a binary vector $g \in \Two^{m_g}$. We can define a state evolution tensor as a map from an action to a state, $\mathbf{T}_g : \Two^n \mapsto \Two^{m_g}$ as,

\begin{equation}
    \mathbf{T}_g \coloneqq (T_g)_{j, k} = g_i \ (X_g)^i_{\; j, k}
\end{equation}

The response of the other player is then $\Sigma^T y$. This yields the evolution,

\begin{equation}
    g^\prime = \mathbf{T}_g (\Sigma^T y)
\end{equation}


\end{document}
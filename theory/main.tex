\documentclass[american]{scrartcl}
    \usepackage{babel}
    \usepackage[utf8]{inputenc} 
    \usepackage{csquotes}
    \usepackage{amsmath}
    \usepackage{amssymb}
    \usepackage{graphicx} 
    \usepackage{subcaption}
    \usepackage{mathtools}
    \usepackage{float}
    \usepackage{fancyvrb} % for "\Verb" macro

    
    \setlength{\parindent}{0em}
    \setlength{\parskip}{0.5em}

    % Bibliography and citations
    \usepackage[bibencoding=utf8, style=apa]{biblatex}
    \bibliography{ref}

    
    \title{A model of Cournot Competition with group selection}


    \author{Andrea Titton}
    

% Commands
\newcommand{\set}[1]{\left\{#1\right\}}
\newcommand{\Real}{\mathbb{R}}
\newcommand{\Rat}{\mathbb{Q}}
\newcommand{\abs}[1]{\left\lvert #1 \right\rvert}
\newcommand{\Two}{\mathbf{2}}
\newcommand{\E}{\mathbb{E}}
\newcommand{\citein}[1]{\citeauthor{#1} (\citeyear{#1})}

\begin{document}

% Title

\maketitle

\section{Introduction}

% FIXME: New interpretation in the introduction


In this short paper I propose to study the dynamics of equilibrium formation in Cournot competition through an evolutionary lens. The aim is to study the formation and sustainability of tacit collusion among firms. I develop a model with local markets competing à la Cournot. Firms evolve in a birth-death process. This can be thought of as firms being able to copy the production technology of their best performing local competitor but having to commit to it for a number of periods. Tacit collusion in this context is defined as an evolution of all quantities produced towards the Cournot equilibrium. Here the standard result applies: tacit collusion is stable only if the number of firms is small (\cite{Lampart2012}).

The model is then extended to encompass group effects, based on the framework by \citein{Akdeniz2020}. In particular, I show how tacit collusion is more likely to be sustained in equilibrium as the probability of outside competition increases.

\section{Literature review}

In 1838, Antoine Augustin Cournot introduced his famous model of competition over quantities. Since then, the model has served as a theoretical benchmark for pure oligopolistic models in economic theory. In Cournot oligopolies, a discrete number of firms compete by setting quantities of a perfectly substitutable good. In equilibrium, due to symmetry, all firms produce the same quantity and any deviations by producers reduces their own profit.

In proposing the model, Cournot does not address the way in which firms form their strategies. Modern economics relies on rational expectations to justify the symmetric equilibrium solution. More contemporary approaches have derived the model dynamics under limited knowledge (\cite{Bischi2015}) or naive expectations (\cite{Cnovas2008}). All of these approaches lead to equilibria that are unstable as the number of firms increases (Cournot–Theocharis problem).

To better model and understand the emergence of such instability I will rely on an evolutionary framework developed by \citein{Akdeniz2020}. The authors analyse the role of group selection in the emergence of cooperation. As in previous models, defection gives a fitness advantage at an individual level (within groups) but a disadvantage at a group level (across groups). Furthermore, groups do not compete globally but only with neighboring groups. The authors show how local competition among groups has strong implications for the emergence of cooperation.

\section{Theoretical formulation}

As in \citein{Akdeniz2020}, the game is played within groups (hereafter, local markets). Firms compete locally à la Cournot. Each period, with some probability, a global event occurs. During a global event, firms of neighboring nodes participate in the game and compete for reproduction with local firms. This mechanism is different from that of group reproduction in the baseline framework (\cite{Akdeniz2020}) because it better encodes the tradeoff between colluding (facing global competition) and competing (getting an edge on local firms) in the Cournot game.

\subsection{Notation}

Any variable $x^{(i)}_t$ is indexed such that the superscript refers to the individual and the subscript to time.

\subsection{Local competition}

Local markets are composed by $N$ firms, indexed by $i$, that can choose a production quantity $q^{(i)} \in \Sigma \subset \Rat_+$. Firms face linear demand,

\begin{equation}
    p(Q) = p\left(q^{(1)}, q^{(2)},  \ldots, q^{(N)} \right) = a - b \cdot \sum^{N}_{i=1} q^{(i)}
\end{equation}

where $Q$ is the vector of quantities, and linear marginal costs,

\begin{equation}
    c(q) = c \cdot q
\end{equation}

All of these assumptions can be easily relaxed to allow for non-linear demand and entry costs. The symmetric Nash equilibrium of the game is (Appendix \ref{A:cournot}),

\begin{equation} \label{equilib}
    \bar{q}= \frac{a - c}{b \cdot (N+1)}
\end{equation}

The model evolves in discrete time $t \in \set{0, 1, \ldots T}$. At $t = 0$, a fixed proportion of firms plays the Cournot equilibrium strategy $\bar{q}$ and the remaining play a strategy sampled from the strategy set, $q^{(i)}_0 \sim U(\Sigma)$. In every period each company realizes a payoff, $\Pi^{(i)}_t = p(Q_t) \cdot q^{(i)}_t - c(q^{(i)})$. I assume that firms reproduce in a birth-death process. That is, each turn, with a probability proportional to its payoffs, a firm is picked to reproduce and, with uniform probability, a firm is picked to die. In particular, the probability of a firm $j$ being picked for reproduction is,

\begin{equation}
    \frac{\Pi^{(j)}}{\sum^n_{i = 1} \Pi^{(i)}}.
\end{equation}

In this setup there is ``evolutionary pressure'' towards higher quantities. To see this consider the setup with only two firms. The best response of each player is the Cournot quantity, which also maximizes total payoffs. But playing a quantity above Cournot reduces the other player's payoffs faster than one own payoffs, which increases the probability of reproduction. The probability of reproduction of a player, given the two quantities played, is plotted in Figure \ref{fig:relprob}. Marked is the Cournot quantity. As the plot illustrates, higher strategies have a higher reproduction probabilities.

\begin{center}
    \begin{figure}[H]
        \center
        \includegraphics[width=0.8\textwidth]{../plots/theory/twoplayers.png}
        \caption{Reproduction probability of player 1 with $N = 2$, $a = 200$, $c = 10$, $b = 1$}
        \label{fig:relprob}
    \end{figure}
\end{center}

This modelling framework is justified in the context of oligopolies where firms copy other firms' production quantities in case the competitor is experiencing better profits. This arises for example in the airline industry, where capacity (number of flights) has to be planned \textit{ex-ante} to comply with regulations. Furthermore, these are markets with high barriers to entry (fixed $N$), long-term investment commitments (randomness in the birth-death process), and common tacit collusion.

\subsection{Global competition}

In the model, there are $M$ local markets and each local market is linked to two other neighboring markets, thus forming a cycle (as in \cite{Akdeniz2020}). Each turn, with probability $\rho$, a local market is involved in a global event. In this case, firms of neighboring markets are involved in the competition for birth. Hence the probability of a firm $j$ from a market $m$ being selected for reproduction is,

\begin{equation}
    \frac{\Pi^{(j)}}{\sum_{i \in (m-1) \cup m \cup (m+1)} \Pi^{(i)}}
\end{equation}

This choice introduces a tradeoff in firms' strategies. Playing a quantity above Cournot increases the firm's payoffs relative to other firms, hence increasing its reproduction probability, but it also reduces the overall payoff of the local market, thereby reducing the reproduction probability in case of a global event. Given this tradeoff we expect that, as the probability of a global event $\rho$ increases, lower quantities (around the Cournot quantity) will be sustained in more markets.

\section{Simulation}

To reproduce the simulation see Appendix \ref{A:code}.

In the simulations I set $a = 200$, $c = 5$, and $b = 1$, such that equation (\ref{equilib}) yields,

\begin{equation}
    \bar{q}(N) = \frac{195}{N+1}
\end{equation}

Furthermore, $N \cdot 0.4$ firms start off playing the Cournot quantity $\bar{q}(N)$ and $N \cdot 0.6$ play a random quantity.

\subsection{Local market}

First we can focus on a simulation of a local market, without group effects (i.e. no global competition). In particular here we look at the evolution of quantities and prices, first with a concentrated market ($N = 5$) and then with a competitive market ($N = 20$). The values of $N$ are set heuristically based on the simulation results.

Figure \ref{fig:small_local} shows a quantity heat map for one of the runs, with $T = 150$, in the case of a concentrated market. Each row represents a firm in the market and each column a time step. Shades of blue (red) represent quantities below (above) Cournot and gray is the Cournot quantity. As expected, quantities converge very rapidly to the Cournot quantity $\bar{q}(5) \approx 81.67$.

\begin{center}
    \begin{figure}[H]
        \center
        \includegraphics[width=0.8\textwidth]{../plots/small/low/localquantity.png}
        \caption{Quantity evolution in a concentrated market}
        \label{fig:small_local}
    \end{figure}
\end{center}

Figure \ref{fig:price_small_local} displays the price evolution $p(Q_t)$ for $100$ periods of $20$ local markets of size $5$. Most markets converge either to the Cournot quantity or above (i.e. display lower prices). This is expected due to the selection towards higher quantities and no outside pressure to collude.

\begin{center}
    \begin{figure}[H]
        \center
        \includegraphics[width=0.8\textwidth]{../plots/small/low/meanprice.png}
        \caption{Price evolution in a concentrated market}
        \label{fig:price_small_local}
    \end{figure}
\end{center}

If we turn our attention to a more competitive markets ($N=20$) and we repeat the exercise, the dynamics change drastically. In particular, Figure \ref{fig:large_local} shows the evolution of strategies of a single run. In this case the market converges immediately to higher quantities due to the increasing competition.

\begin{center}
    \begin{figure}[H]
        \center
        \includegraphics[width=0.8\textwidth]{../plots/big/low/localquantity.png}
        \caption{Quantity evolution in a competitive market}
        \label{fig:large_local}
    \end{figure}
\end{center}

The lack of collusion can be seen more clearly in the 150 simulations run of price in Figure \ref{fig:price_large_local}.

\begin{center}
    \begin{figure}[H]
        \center
        \includegraphics[width=0.8\textwidth]{../plots/big/low/meanprice.png}
        \caption{Price evolution in a competitive market}
        \label{fig:price_large_local}
    \end{figure}
\end{center}

\subsection{Global markets}

To study the effect of groups on the equilibria of the game we first look at a simple simulation run with, as before, $N = 5$ and $M = 20$. In this case, $\rho = 1$, such that every period there is a global event in each group. In Figure \ref{fig:cert}, I plot the price evolution for $N = 5$ (\ref{fig:cert:small}) and $N = 20$ (\ref{fig:cert:big}). In both cases the effect of global competition on prices is evident, namely outside pressure induces tacit collusion by reducing the advantage of playing higher quantities. With small $N$, all local markets converge to the Cournot price. With a high $N$, 150 time points are not sufficient to achieve stability but the density of local markets playing a quantity higher than Cournot decreases visibly.

\begin{figure}[H]
    \begin{subfigure}{.5\textwidth}
        \centering
        \includegraphics[width=\textwidth]{../plots/small/high/meanprice.png}
        \caption{$N = 5$}
        \label{fig:cert:small}
    \end{subfigure}%
    \begin{subfigure}{.5\textwidth}
        \centering
        \includegraphics[width=\textwidth]{../plots/big/high/meanprice.png}
        \caption{$N = 20$}
        \label{fig:cert:big}
    \end{subfigure}
    \caption{Simulation with $\rho = 1$}
    \label{fig:cert}
\end{figure}

To better understand the effect of an increase in $\rho$, I simulated 150 runs with $T = 150$ and $M = 20$ for each value of $\rho \in \set{0, 0.01, 0.02, \ldots 0.99, 1}$. In Figure \ref{fig:sim} I plot the average and standard deviation of prices at time 150 of all groups. As expected, both simulations, with $N = 5$ (\ref{fig:sim:small}) and $N = 20$ (\ref{fig:sim:big}), show a clear effect increase in prices as $\rho \xrightarrow{} 1$. Furthermore, concentrated markets display higher prices than competitive markets for every value of $\rho$.

\begin{figure}[H]
    \begin{subfigure}{.5\textwidth}
        \centering
        \includegraphics[width=\textwidth]{../plots/small/mean_price_sim.png}
        \caption{$N = 5$}
        \label{fig:sim:small}
    \end{subfigure}%
    \begin{subfigure}{.5\textwidth}
        \centering
        \includegraphics[width=\textwidth]{../plots/big/mean_price_sim.png}
        \caption{$N = 20$}
        \label{fig:sim:big}
    \end{subfigure}
    \caption{Long term price convergence}
    \label{fig:sim}
\end{figure}

\section{Conclusions}

In this short paper I present a tentative analysis of the dynamics of an oligopolistic market with Cournot competition and group structure. First, the familiar result of Cournot equilibria being stable only in small markets applies. In small markets and without group effects the average price across local markets converges quickly towards the Cournot price while the average price in large markets converges towards smaller prices. Second, I show how group structure, in the form of competition from outside the local market, can reduce competition and induce a collusive equilibrium. Such an effect is present both in small and big local markets.

This exercise is by no means exhaustive but it highlights one possible channel that could explain the persistence of tacit collusion in large markets. Furthermore, it shows the descriptive potential of observing the dynamics of economic phenomena through an evolutionary lense.

\newpage
% Bibliography
\nocite{*}
\pagenumbering{gobble} % stop page numbering
\printbibliography

\newpage
\appendix

\section{Cournot equilibrium} \label{A:cournot}

Let $q^{(-i)} = Q \setminus \set{i}$. Then the payoff function of firm $i$ is,

\begin{equation*}
    \Pi\left(q^{(i)}, q^{(-i)} \right) = q^{(i)} \cdot p(Q)
\end{equation*}

Then the best response strategy, under ration expectations, is,

\begin{equation*}
    \begin{split}
        \bar{q} = \arg\max_{q} \Pi\left(q, q^{(-i)} \right),
    \end{split}
\end{equation*}

which yields the first order condition,

\begin{equation*}
    \bar{q} \cdot \frac{ \partial p}{\partial q}\left(\bar{q}, q^{(-1)}\right) + p\left(\bar{q}, q^{(-1)}\right) = 0.
\end{equation*}

With a linear demand function, by symmetry it is straight forward to show that,

\begin{equation*}
    \bar{q} = \frac{a}{b \cdot (N+1)}
\end{equation*}


\section{Simulation code} \label{A:code}

The code can be found at,

\begin{verbatim}
    github.com/NoFishLikeIan/tinbergen/tree/master/code/evolutionary
\end{verbatim}

The simulation requires \verb+Julia v.1.5.x > +. A simulation to produce all graphs is available as,

\begin{verbatim}
    julia main.jl
\end{verbatim}

and to run the coalition size simulation as,

\begin{verbatim}
    julia coalitionsize.jl
\end{verbatim}




\end{document}